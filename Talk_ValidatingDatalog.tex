\documentclass[aspectratio=169]{beamer}
\usepackage[utf8x]{inputenc}
\usepackage{amssymb}
\usepackage{appendixnumberbeamer}
\usepackage{pgfplots}
\usepackage{array}
\usepackage{tikz}
\usetikzlibrary{automata, positioning, chains,fit,shapes}
\usepackage{color}
\definecolor{keywordcolor}{rgb}{0.7, 0.1, 0.1}   % red
\definecolor{commentcolor}{rgb}{0.4, 0.4, 0.4}   % grey
\definecolor{symbolcolor}{rgb}{0.0, 0.1, 0.6}    % blue
\definecolor{sortcolor}{rgb}{0.1, 0.5, 0.1}      % green
\usetheme{Dresden}
\usepackage{listings}
\def\lstlanguagefiles{lstlean.tex}
\lstset{
    language=lean,
}
\setbeamercolor{block body alerted}{bg=alerted text.fg!10}
\setbeamercolor{block title alerted}{bg=alerted text.fg!20}
\setbeamercolor{block body}{bg=structure!10}
\setbeamercolor{block title}{bg=structure!20}
\setbeamercolor{block body example}{bg=green!10}
\setbeamercolor{block title example}{bg=green!20}

\title[Datalog validation]{Validation of datalog reasoning results}
\author[J. Tantow]{Johannes Tantow}
\setbeamertemplate{navigation symbols}{%
    \usebeamerfont{footline}%
    \usebeamercolor[fg]{footline}%
    \hspace{1em}%
    \insertframenumber/\inserttotalframenumber
}
\setbeamercolor{footline}{fg=black}
\setbeamerfont{footline}{series=\bfseries}
% Custom command for the arrow

% Define a custom color for the arrow
\definecolor{arrowcolor}{RGB}{0, 102, 204}

% Custom command for the arrow
\newcommand{\arrowlong}[1]{%
    \tikz[baseline=-0.5ex]\node[fill=arrowcolor, single arrow, shape border rotate=0, text=white, draw=arrowcolor, inner sep=2pt, minimum height=1cm] {#1};%
}


\begin{document}
    \maketitle

    \begin{frame}[fragile]{Datalog}
        \begin{columns}[t]
            \begin{column}{0.4\textwidth}
                \begin{lstlisting}[basicstyle=\small]
a(1,2).
b(3,4).

c(?x,?y) :- a(?x,?y).

d(?x,?y) :- b(?x,?y).

e(?x,?y) :- c(?x,?z), d(?a, ?y).
                \end{lstlisting}
            \end{column}
            \begin{column}{0.25\textwidth}
                \arrowlong{Datalog engine}

            \end{column}
            \begin{column}{0.25\textwidth}
                
                \begin{lstlisting}[basicstyle=\small]
a(1,2).
b(3,4).

c(1,2) :- a(1,2).

d(3,4) :- b(3,4).

e(1,4) :- c(1,2), d(3, 4).

----
c(1,3) :- a(1,3).

                                    \end{lstlisting}
                
                                
            \end{column}
        \end{columns}
        
    \end{frame}

    \begin{frame}[fragile]{Proof trees}
        \begin{columns}
            \begin{column}{0.3\textwidth}

       
                \begin{tikzpicture}[every node/.style={circle, minimum size=8mm}, level/.style={sibling distance=40mm/#1}, level distance=20mm]
                    \node {e(1,4)}
                        child {node {c(1,2)}
                            child {node {a(1,2)}}
                        }
                        child {node {d(3,4)}
                            child {node {b(3,4)}}
                        };
                \end{tikzpicture}
            \end{column}
            \begin{column}{0.5\textwidth}
                \begin{lstlisting}
def isValid(P: program τ) (d: database τ) (t: proofTree τ): Prop :=
  match t with
  | proofTree.node a l => 
  ( ∃(r: rule τ) (g:grounding τ),
    r ∈ P ∧ ruleGrounding r g = groundRuleFromAtoms a (List.map root l) 
    ∧ l.attach.All₂ (fun ⟨x, _h⟩ => isValid P d x)) 
  ∨ (l = [] ∧ d.contains a)
                \end{lstlisting}
            \end{column}
        \end{columns}
    \end{frame}

    \begin{frame}[fragile]{Datalog modelling}
        \begin{columns}
            \begin{column}{0.4\textwidth}
                Syntax:

                \begin{lstlisting}
structure rule where
(head: atom τ)
(body: List (atom τ))

def grounding:= 
    τ.vars → τ.constants

                \end{lstlisting}
                
            \end{column}
            \begin{column}{0.5\textwidth}
                Semantics:

                \begin{lstlisting}[basicstyle=\small\ttfamily]
def proofTheoreticSemantics (P: program τ) (d: db):= 
{a:| ∃ (t: proofTree τ), root t = a ∧ isValid P d t}

theorem SemanticsEquivalence 
 (P: program τ) (d: db τ): proofTheoreticSemantics P d = modelTheoreticSemantics P d :=


                \end{lstlisting}
                
            \end{column}
        \end{columns}

    \end{frame}
    \begin{frame}[fragile]{Tree validation}
        \begin{columns}
            \begin{column}{0.35\textwidth}
                \begin{lstlisting}
                    
P:

r(x,y) :- s(y,x).

r(x,y) :- s(x,z),t(z,y).
                \end{lstlisting}

            \end{column}

            \begin{column}{0.4\textwidth}
                \begin{tikzpicture}[every node/.style={circle, minimum size=8mm}, level/.style={sibling distance=40mm/#1}, level distance=20mm]
                    \node {r(a,b)}
                        child {node {s(a,b)}
                        }
                        child {node {t(b,c)}
                        };
                \end{tikzpicture}

                Formally implemented and proved unification:

                Try:
                
                r(x,y) :- s(x,z),t(z,y).


                with
                
                r(a,b) :- s(a,b), t(b,c)
                
            \end{column}
        \end{columns}
    \end{frame}
    \begin{frame}[fragile]{Completeness}
        Is our solution all we can derive ?
        
        i elements in valid trees

        $i \subseteq \mathtt{proofTheoreticSemantics\ P\ d} = \mathtt{modelTheoreticSemantics\ P\ d} \subseteq i$

        \begin{tikzpicture}[
            every node/.style={circle, minimum size=8mm},
            level/.style={sibling distance=20mm/#1, level distance=20mm},
            grow'=right, % Set the tree growth direction to right
            anchor=west % Anchor the tree at the west (left) side
        ]
            \node {r(x,y) :- s(y,x)}
                child {node {r(a,y) :- s(y,a)}
                }
                child {node {r(1,y) :- s(y,1)}
                    child {node {r(1,2) :- s(3,1)}}
                    child {node {r(1,3) :- s(3,1)}}
                };
        \end{tikzpicture}
    \end{frame}
\end{document}